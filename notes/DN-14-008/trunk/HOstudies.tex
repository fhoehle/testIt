\section{Studies with the Hadron Outer Calorimeter}
In the barrel region of the CMS experiment the hadronic calorimeter has a outer component placed just behind the solenoid and before the first muon stations. This subdetector is called hadron outer
(HO) calorimeter and it is a tail catcher for jets leaking out of the inner hadronic calorimeter. In this section first of all the HO system and its readout structure are introduced. Then some studies
on detection efficiency of muons using 2012 data are shown. The analysis of the detection efficiency for cosmic muons from GRIN data is conclusively dealt with.
	\subsection{The Hadron Outer (HO) Calorimeter}
		to do with Andreas
  		\subsubsection{Benefits and constraints}
			to do with Andreas
  		\subsubsection{Design of the system}
			to do with Andreas
	\subsection{Readout logic and DIGI structure}
		to do with Andreas
  		\subsubsection{Readout setup}
			to do with Andreas
	\subsection{Studies on detection efficiency of prompt muons using 2012 data}
		Due to the similarity of the setup of HO and MTT we expect to find answers to some open questions concerning the MTT concept like the muon detection capability of a
		scintillator system e.g. Therefore the detection efficiency for tight ID muons in HO from 2012 data has been studied. For this purpose the muons have to fulfill some selection createria and they
		have to be accepted by the HO system.
		\subsubsection{Muon selection and acceptance by HO}
			todo
		\subsubsection{Matching of the muons to the corresponding HO tiles using standard tracking tools}
			todo
		\subsubsection{Detection efficiency for prompt muons}
			todo
	\subsection{Studies on detection efficiency of cosmic muons using the GRIN data} 
		todo
		\subsubsection{Detector setup for the Global Run In November (GRIN)}
			todo
		\subsubsection{Purity studies}
			todo
		\subsubsection{Efficiency studies}
			todo
		\subsubsection{Working point for triggering muons}
			todo
