\section{Muon ghosts}
\subsection{The DT System}
\label{DTSystem}
The drifttube (DT) system of the muon detector at CMS consists of several organization structures.\\
The smallest structure, the drift tubes are organized within a superlayer (SL). A superlayer consists of four layers of DTs.\\
Every sector per wheel in CMS has four muon stations (counted 1 to 4 beginning with the most inner sation).\\
A muon station is the composition of two SLs that are oriented in phi direction (stations 1-4) and one SL in theta direction (stations 1-3). The hardware of one station is assembled in a minicrate (MC).\\
The trigger chain of the DT system starts within a SL. Nine of the DTs are combined to a bunch and track identifier (BTI) trigger. BTI Trigger have an overlap region to avoid inefficiencies and obtain redundancy. If three of four DT see a signal for a certain bunch crossing a low quality trigger (LQTR) is generated. If four DTs in line see a signal a high quality trigger (HQTR) is generated. For now only HQTR are processed to the next trigger step.\\
The track correlator (TRACO) combines the HQTR of the two BTI of the SL in phi direction within one station. TRACO triggers can be generated for mu with an angle of XXX and less.\\
These TRACO triggers are transmitted via a 40 lines flat cable to the trigger server (TS). The TS is located on the MC of station 4.\\
The TS has a phi and a theta part. The TS phi correlates the TRACO tracks in phi direction. The TS theta correlates the tracks in the theta view.\\
This data is transmitted via Low Voltage Differential Signaling (LVDS) to the sector collector (SC) that is located in the service cavern (USC).\\
From there the signal is processed to the global muon trigger (GMT) and the global trigger (GT).
For the setup of the muon trigger system, see also TODO ref TDR Muon.
\subsection{Reconstruced Ghosts}
\subsubsection{Ghost definition}
Ghosts are faked muons that are detected by the muon system although there is no physical muon present.
Muon ghosts can occur in all stages of the detection and reconstruction system of CMS beginning with the low level triggers up to the final reconstructed data (TODO ref Bild Einfuehrungskapitel).
A ghost can also be a detected muon that has the wrong time stamp and thus is assigned to the wrong bunch crossing.

\subsubsection{Reco ghosts}
Following the definition of ghosts, a RECO ghost is a reconstructed muon that has no matching generated (GEN) muon.\\
For this matching for every RECO mu the nearest GEN mu in terms of delta R is determined. Additionally, the vertex position of the GEN and RECO mu are not allowed to be more than 0.1 cm away from each other in every of the three coordinates to assure that the RECO mu is not from a pileup event.\\
The following investigations are performed with global muons only to have muon system and tracker information available for every muon. Furthermore the eta direction of the investigated muons is cut to be abs(eta) < 0.8 to check for muons in the barrel region only. The sample is a ZZ$\to$4l with 2012 conditions.\\
$\Delta R$ is plotted for all GEN muons with their matched RECO muon. If two RECO muon match to the same GEN muon, $\Delta R$ is plotted for both RECO muons.\\
The $\Delta R$ distribution is shown in figure \ref{DeltaRDistribution}. The most matched muons have a very small $\Delta R$. Only a few muons have a $\Delta R$ above 0.1 with a slight maximuim at 1.5. It is obvious that the most muon combinations with $\Delta R > 0.1$ come from muons in ghost events. Therefore, a separation between ghosts and real muons using $\Delta R$ seems to be feasible and is discussed.\\
\begin{figure}[b]
\centering
\begin{minipage}[t]{0.95\textwidth}
\includegraphics[width=\textwidth]{Figures/scheuch/Trennung.png}
\caption{$\Delta R$ between all RECO muons and the matched GEN muon in all events and in events that are marked as ghost events}
\label{DeltaRDistribution}
\end{minipage}
\end{figure}
If one GEN mu has two or more associated RECO mu all but one of these RECO mu are considered to be ghosts. An event with at least one ghost is named ghost event.\\
For the separation if a RECO mu is a ghost or real, a delta R cut is introduced. If the delta R between RECO mu and associated GEN mu is smaller than 0.1, the mu is considered to be a correctly reconstructed mu. If the delta R is greater than 0.1, it is considered to be a ghost.

\subsubsection{Matching efficiency and quality of separation}
To determine if the matching algorithm is correct and efficient some control variables have been observed.\\
The distance of vertex position in the z direction (meaning along the beam axis) of the RECO mu and the associated GEN mu has been plotted (see figure \ref{DeltaZMatching}). The distance is within the resolution of the tracker system. Nevertheless, all muons with a distance of more than 0.1, since for large distances in the z vertex a mismatching can not be imposed.\\
\begin{figure}[b]
\centering
\begin{minipage}[t]{0.95\textwidth}
\includegraphics[width=\textwidth]{Figures/scheuch/DeltaZ.png}
\caption{Distance of the z position of the vertex RECO muon and the matched GEN muon}
\label{DeltaZMatching}
\end{minipage}
\end{figure}
To show, that the determination between ghost and real muon is reliable, the pt ratio between GEN mu and RECO mu is plotted for identified ghosts (see figure \ref{PtRatioGhost}) and real mu within a ghost event (see figure \ref{PtRatioReal}).
\begin{figure}[b]
\centering
\begin{minipage}[t]{0.475\textwidth}
\includegraphics[width=\textwidth]{Figures/scheuch/ptRatioRealMu.png}
\caption{$\mathrm{p{T}}$ ratio of the GEN and RECO mu for identified real RECO muons}
\label{PtRatioReal}
\end{minipage}
\hspace{0.5cm}
\begin{minipage}[t]{0.475\textwidth}
\includegraphics[width=\textwidth]{Figures/scheuch/ptRatioGhosts.png}
\caption{$\mathrm{p{T}}$ ratio of the GEN and RECO mu for identified RECO ghosts}
\label{PtRatioGhost}
\end{minipage}
\end{figure}
The pt ratio peaks at 1 for real mu. This indicates, that the RECO mu is in fact the correct reconstruction of the generated particle. On the other hand, the pt ratio of the ghost mu shows a non peaking shape. Therefore, it can be assumed that the pt of the ghosts is not related to the matched GEN particle. Furthermore, the pt of the ghost is smaller than the pt of the generating particle in the majority of ghost events.\\
Furthermore, under the assumption that there is one ghost per real muon only in the majority of the events one expects that the number of ghosts and real muons is equal for a certain GEN muon. Comparing the number of ghosts (12175) and the number of real muons (12013) shows that the seperation between the real and the ghost muon is working well.

\subsubsection{Ghost busting with HO information}
To determine if hadron outer (HO) information can be used to bust ghosts on L1 Level the calorimeter information is used for 2012 simulations, since HO did not work properly during this phase. In a first step, the likelihood ratio of the calorimeter system (ECAL, HCAL, HO) defined as
\begin{equation}
L=\frac{L_{\mathrm{muon}}}{L_{\mathrm{muon}} + L_\mathrm{not\ muon}}
\end{equation}.
is plotted.
\begin{figure}[b]
\centering
\begin{minipage}[t]{0.95\textwidth}
\includegraphics[width=\textwidth]{Figures/scheuch/LikelihoodNonGhost.png}
\caption{Likelihood ratio for an MIP like entry in the calorimeter system for real muons}
\label{LikelihoodReal}
\end{minipage}
\end{figure}
\begin{figure}
\begin{minipage}[t]{0.95\textwidth}
\includegraphics[width=\textwidth]{Figures/scheuch/LikelihoodGhost.png}
\caption{Likelihood ratio for an MIP like entry in the calorimeter system for ghosts}
\label{LikelihoodGhost}
\end{minipage}
\end{figure}
For identified muons the likelihood ratio peaks at 1. This is consistent with the fact that real muons should deposit energy within the calorimeter system.\\
The likelihood ratio for ghosts peaks at 0, has a miminum at 0.4 and rises again up to 1 without reaching a maximum at 1. The ghosts should not deposit energy in the calorimeter system. For some of the ghosts this seems to be the case. For the other ghosts the likelihood might be tending to 1 because energy is deopsited in the calorimeter system by other systems oder due to some noise. This can also explain why the likelihood ratio drops at the 1 bin for ghosts.\\
In a next step, the energy information in the HO system for an HO tower has to be investigated to obtain, if HO can serve as muon veto and trigger. This has to be redone also for HL-LHC conditions. Here one has to check if the granularity of HO is sufficient to deal with the additional pile up and boosted events.
\subsection{BTI Ghosts}
\subsubsection{Ghost definition}
A BTI ghost is, if only one muon passes a muon station and two BTI HQTR are created in this station. (For the description of the BTI system see chapter \ref{DTSystem}.)
\subsubsection{Investigation of the BTI Trigger}
To see if the BTI trigger experiences ghost phenomena a muon gun study was performed. Events without pileup have been created. Every event contains one muon that is passing directly through on of the 12 sectors. This muon hits all four muon chambers. A second muon was shot through a sector next to the sector of the first muon. The number of BTI triggers in the first station were counted.\\
The efficiency is defined as
\begin{equation}
E=\frac{\sum_{i = 1}^\infty N_{\mathrm{BTI}_i}}{\sum_{i = 0}^\infty N_{\mathrm{BTI}_i}}
\end{equation}.\\
An effiency of about 80\% is reached for one station only (see figure \ref{BTIEfficiency}).
\begin{figure}
\begin{minipage}[t]{0.95\textwidth}
\includegraphics[width=\textwidth]{Figures/scheuch/SectorGunPt100dPhi0_5Phi0_h1dFilteredBtiHitsPerEvtSL1.png}
\caption{Number of BTI trigger in station one for one passing muon and BTI trigger efficiency}
\label{BTIEfficiency}
\end{minipage}
\end{figure}
The number of events with BTI ghosts is defined as
\begin{equation}
N_{\mathrm{Ghost\ Event}} = \sum_{i = 2}^\infty N_{\mathrm{BTI}_i}
\end{equation}.
This number goes up to 5 if no filter on the bunchcrossing ID is used. The filter on the bunchcrossing ID can reduce this number of BTI ghosts significanly. Using the filter on the bunchcrossing ID the ghost rate at BTI level is reduced to about 1\%.\\
Unfortunately, an adjustment oder modification of the BTI system is not possible for the planned upgraded phases since it would need direct access to every muon station. Therefore, the number of BTI trigger can not be reduced by HO like systems.