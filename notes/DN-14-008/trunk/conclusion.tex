\section{Conclusion}
	In the CMS barrel region the outer layers of the Hadron Calorimeter (HO) were successfully upgraded to SiPM readout during Long Shutdown 1. 
	The SiPM signals could be used in the muon trigger at Level 1.
	The benefits of this detector layer for MIP identification, punchthrough rejection and resolution of ambiguities in the DT system are
	currently being studied with simulations.
	For Run II a trigger link is currently being established that makes the HO signals available for building Level 1 muon trigger primitives together with inputs from DT and RPC. Based on these data,
	one could evaluate whether to optimize the HO granularity for Level 1 muon triggering purposes in Phase 2, in combination with the barrel muon track finder and Track Trigger systems.
	With the integration of a simple MTT system in CMSSW a tool is at hand that can be used in combination with the outcome of the HO studies at Run II to optimize the granularity.