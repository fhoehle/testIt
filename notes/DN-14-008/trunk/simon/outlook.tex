\newpage
\subsection{Upcoming prototype studies}
\label{section:simon/outlook}
The presented studies show that the $100 \times 100 \times 5 \,\text{mm}^3$ MTT prototype can be used as a reliable muon detector. Further studies with this prototype will investigate the timing resolution of the module, which is an important attribute for a trigger detector. In addition the homogeneity of the signal height as well as the homogeneity of the timing resolution will be determined. Also Tyvek paper is being considered as an alternative for the PTFE tape as diffuse reflector.

Since the optimal granularity of a possible MTT system has yet do be identified by simulations, larger prototypes will be studied as well. Therefore prototypes with $300 \times 300 \times 5 \,\text{mm}^3$ scintillators have already been produced, some of which with integrated wavelength shifting fibres. All prototypes will be tested in a proton test beam at Forschungszentrum J\"ulich in September 2014. With the planned setup it will be possible to investigate how the average signal height, the detection efficiency and the timing resolution are distributed over the detector surface and how different fibre layouts compare to each other in the larger modules.
