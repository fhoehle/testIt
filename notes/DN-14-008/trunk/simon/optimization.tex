\subsection{Parameter optimization}
\label{section:simon/optimization}
To achieve the best possible light yield and gain a deeper understanding of the prototype several parameters have been investigated separately, some of which are presented in this section. To compare different setups, each configuration has been tested with 3000 -- 5000 cosmics and a landau fit was performed on the resulting QDC or FADC spectrum (as previously shown in Figure \ref{image:Figures/simon/setup/Wrapping2_PULSE_S_FIT.pdf}). The most probable values (MPV) of these fits are used as a quality criteria for the given configuration and presented for all three signal outputs (SiPM A, analogue sum, SiPM B) in bar charts.
%
\subsubsection{Wrapping material}
A very important component of the detector is the reflective wrapping material of the scintillator since it directly effects the light yield. Therefore aluminium foil has been tested as a specular reflector and compared to a multi layer wrapping of PTFE tape as a diffuse reflector. The result is shown in Figure \ref{image:Figures/simon/optimization/VergleichAluTape_QDC.pdf}. Wrapping the scintillator with PTFE tape leads to a significant increase in the signal height of about $70\,\%$ -- $90\,\%$. Therefore, all further measurements have been performed with PTFE tape as reflector, applied in three layers.
\image{width = .6\textwidth}{Figures/simon/optimization/VergleichAluTape_QDC.pdf}{Comparison between aluminium foil and PTFE tape as reflective wrapping material}
%
\subsubsection{Scintillator thickness}
Since the CMS detector is by definition a very compact experiment, one of the goals for the MTT system is to provide a thin detector layer. Therefore the effect of the scintillator thickness is of interest and a comparison between a $5\,\text{mm}$ and a $10\,\text{mm}$ thick scintillator has been made. The results of this comparison are depicted in Figure \ref{image:Figures/simon/optimization/VergleichDicke_QDC.pdf}. The light yield drops when changing the $10\,\text{mm}$ scintillator to a $5\,\text{mm}$ thick one, but not as much as one would expect. The signal loss lies in the region of $10\,\%$ -- $15\,\%$, which is an acceptable trade off for a $50\,\%$ decrease in detector thickness. Therefore, all further measurements have been performed with a $5\,\text{mm}$ thick scintillator.
\image{width = .6\textwidth}{Figures/simon/optimization/VergleichDicke_QDC.pdf}{Comparison between a $5\,\text{mm}$ and a $10\,\text{mm}$ thick scintillator}
%
\subsubsection{Coupling between SiPM and scintillator}
\doubleimage 
{width = .4\textwidth}{Figures/simon/optimization/coupling/grease.jpg}{Silicone grease: Saint-Gobain BC-630}
{width = .4\textwidth}{Figures/simon/optimization/coupling/grease_dry.JPG}{Dried BC-630}
{The silicone grease BC-630 by Saint-Gobain dries within a couple of days}
{.07\textwidth}
One of the most critical points of scintillator based detectors is the optical coupling between the scintillator and the light detector. In the first MTT prototypes, optical coupling has been done with the \emph{BC-630 optical grease} by Saint-Gobain which is shown in Figure \ref{image:Figures/simon/optimization/coupling/grease.jpg} a). After applying the grease, the detector signals dropped significantly within a week's time, which can be explained with Figure \ref{image:Figures/simon/optimization/coupling/grease.jpg} b). The figure shows the scintillator edge to which the grease had been applied. One can clearly make out that the grease has dried out or partially evaporated on the surface and therefore the optical connection between the SiPM and the scintillator suffered significantly. With this result, the BC-630 can obviously not be considered as a long solution for the MTT system. Therefore two other methods of coupling the SiPM to the scintillator have been tested: The silicone gel \emph{RTV 6151} by GE Silicones and silicone rubber pads made out of \emph{RT 604} by ELASTOSIL. Pictures of both options are depicted in Figure \ref{image:Figures/simon/optimization/coupling/silicone_gel.jpeg}.
\doubleimage
{width = .4\textwidth}{Figures/simon/optimization/coupling/silicone_gel.jpeg}{Silicone gel: GE Silicones RTV 6151}
{width = .4\textwidth}{Figures/simon/optimization/coupling/silicone_rubber.jpg}{Silicone rubber: ELASTOSIL RT 604}
{The two component silicone gel RTV 6151 by GE Silicones and silicone rubber pads made out of ELASTOSIL RT 604 were tested as alternatives}
{.07\textwidth}
Both alternatives are silicone based, mixed of two components and deaerated under a vacuum bell jar. The RTV 6156 has been applied directly to the scintillator, cured in the assembled module and therefore adopted perfectly to the SiPM geometry. On the other hand, a $1\,\text{mm}$ thick layer of RT 604 had been cast and hardend on a glass surface from which small pads can be cut out. The idea to produce these pads is inspired by conventional photo multiplier tubes, where similar pads have been used for many years as optical coupling between scintillators and light guides. The great advantage of these pads is their reusability and the simple detector assembly.

Measurements with all these coupling methods including one without any additional optical layer have been performed with cosmic muons. The results are shown in Figure \ref{image:Figures/simon/optimization/VergleichKopplungen_PULSE.pdf}. The importance of an optical medium is clearly visible, all three methods lead to an significant increase of light yield. Out of the three substances the BC-630 provided the best result, even though it cannot be used long term as described above. The other two methods are slightly but not significantly worse. The RTV 6151 should be used as the long term solution for the MTT system. However, for testing purposes the RT 604 pads have been chosen for all following measurements due to their flexibility and ease of use. 
\image{width = .6\textwidth}{Figures/simon/optimization/VergleichKopplungen_PULSE.pdf}{Comparison between different optical couplings between the SiPMs and the scintillator}
%
%
\subsubsection{Coupling reproducibility}
\label{section:simon/optimization/recoupling}
Because the prototype module often had to be reassembled for different setups, it is important to investigate how this effects the detector signals. For this purpose the frontend electronics have been separated from the module and reassembled with new silicone pads four times. The results depicted in Figure \ref{image:Figures/simon/optimization/VergleichZusammenbau_PULSE.pdf} show uncritical fluctuations in the region of $\leq 5\,\%$ (RMS). 
\image{width = .6\textwidth}{Figures/simon/optimization/VergleichZusammenbau_PULSE.pdf}{Effect of reassembly of the module using silicone rubber}
%
\subsubsection{Wrapping reproducibility}
\label{section:simon/optimization/rewrapping}
Similar to the investigation of the coupling reproducibility, the wrapping of the scintillator with PTFE tape has been tested by means of rewrapping the whole scintillator four times. In addition a new wrapping technique with particular attention to the scintillator edges and the reduction of possible air pockets has been applied. The results are shown in Figure \ref{image:Figures/simon/optimization/VergleichWrapping_PULSE.pdf}. The new wrapping technique clearly increases the light yield, which was up to $30\,\%$ lower with the old wrapping. The fluctuations of the signal height when rewrapping the scintillator are about $8\,\%$ (RMS).
\image{width = .6\textwidth}{Figures/simon/optimization/VergleichWrapping_PULSE.pdf}{Effect of rewrapping the scintillator tile with PTFE tape}