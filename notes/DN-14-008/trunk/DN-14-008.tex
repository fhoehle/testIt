% Customizable fields and text areas start with % >> below.
% Lines starting with the comment character (%) are normally removed before release outside the collaboration, but not those comments ending lines

% svn info. These are modified by svn at checkout time.
% The last version of these macros found before the maketitle will be the one on the front page,
% so only the main file is tracked.
% Do not edit by hand!
\RCS$Revision: 240525 $
\RCS$HeadURL: svn+ssh://svn.cern.ch/reps/tdr2/notes/DN-14-008/trunk/DN-14-008.tex $
\RCS$Id: DN-14-008.tex 240525 2014-05-06 12:16:46Z alverson $
%%%%%%%%%%%%% local definitions %%%%%%%%%%%%%%%%%%%%%
% This allows for switching between one column and two column (cms@external) layouts
% The widths should  be modified for your particular figures. You'll need additional copies if you have more than one standard figure size.
\newlength\cmsFigWidth
\ifthenelse{\boolean{cms@external}}{\setlength\cmsFigWidth{0.85\columnwidth}}{\setlength\cmsFigWidth{0.4\textwidth}}
\ifthenelse{\boolean{cms@external}}{\providecommand{\cmsLeft}{top}}{\providecommand{\cmsLeft}{left}}
\ifthenelse{\boolean{cms@external}}{\providecommand{\cmsRight}{bottom}}{\providecommand{\cmsRight}{right}}
%%%%%%%%%%%%%%%  Title page %%%%%%%%%%%%%%%%%%%%%%%%
\cmsNoteHeader{DN-14-008} % This is over-written in the CMS environment: useful as preprint no. for export versions
% >> Title: please make sure that the non-TeX equivalent is in PDFTitle below
\title{A new trigger concept for muons in the barrel region of the CMS experiment: Muon Track fast Tag}

% >> Authors
%Author is always "The CMS Collaboration" for PAS and papers, so author, etc, below will be ignored in those cases
%For multiple affiliations, create an address entry for the combination
%To mark authors as primary, use the \author* form
\address[neu]{RWTH Aachen University}
\author[neu]{Yusuf Erdogan}
\author[neu]{G\"unter Fl\"ugge}
\author[neu]{Thomas Hebbeker}
\author[neu]{Andreas K\"unsken}
\author[neu]{Erik Dietz-Laursson}
\author[neu]{Markus Merschmeyer}
\author[neu]{Oliver Pooth}
\author[neu]{Thomas Rademacher}
\author[neu]{Florian Scheuch}
\author[neu]{Achim Stahl}
\author[neu]{Simon Weingarten}
\author[neu]{Lars Weinstock}

% >> Date
% The date is in yyyy/mm/dd format. Today has been
% redefined to match, but if the date needs to be fixed, please write it in this fashion.
% For papers and PAS, \today is taken as the date the head file (this one) was last modified according to svn: see the RCS Id string above.
% For the final version it is best to "touch" the head file to make sure it has the latest date.
\date{\today}

% >> Abstract
% Abstract processing:
% 1. **DO NOT use \include or \input** to include the abstract: our abstract extractor will not search through other files than this one.
% 2. **DO NOT use %**                  to comment out sections of the abstract: the extractor will still grab those lines (and they won't be comments any longer!).
% 3. For PASs: **DO NOT use tex macros**         in the abstract: CDS MathJax processor used on the abstract doesn't understand them _and_ will only look within $$. The abstracts for papers are hand formatted so macros are okay.
\abstract{
   This is an example of a \textit{CMS Note} written in \LaTeX
    using the \emph{cms-tdr} document class and processed using the
    same \texttt{tdr} perl script used in generating the CMS Physics TDRs.
    Instructions for producing CMS Notes and Internal Notes are given.
}

% >> PDF Metadata
% Do not comment out the following hypersetup lines (metadata). They will disappear in NODRAFT mode and are needed by CDS.
% Also: make sure that the values of the metadata items are sensible and are in plain text:
% (1) no TeX! -- for \sqrt{s} use sqrt(s) -- this will show with extra quote marks in the draft version but is okay).
% (2) no %.
% (3) No curly braces {}.
\hypersetup{%
pdfauthor={George Alverson, Lucas Taylor, A. Cern Person},%
pdftitle={A new trigger concept for muons in the barrel region of the CMS experiment: Muon Track fast Tag},%
pdfsubject={CMS},%
pdfkeywords={CMS, physics, software, computing}}

\maketitle %maketitle comes after all the front information has been supplied
% >> Text
%%%%%%%%%%%%%%%%%%%%%%%%%%%%%%%%  Begin text %%%%%%%%%%%%%%%%%%%%%%%%%%%%%
%% **DO NOT REMOVE THE BIBLIOGRAPHY** which is located before the appendix.
%% You can take the text between here and the bibiliography as an example which you should replace with the actual text of your document.
%% If you include other TeX files, be sure to use "\input{filename}" rather than "\input filename".
%% The latter works for you, but our parser looks for the braces and will break when uploading the document.
%%%%%%%%%%%%%%%

\input{HOStudies.tex}
\input{MTTSimulation.tex}

% >> acknowledgements (for journal papers)
% Please include the latest version from https://twiki.cern.ch/twiki/bin/viewauth/CMS/Internal/PubAcknow.
%\section*{Acknowledgements}
% ack-text

%% **DO NOT REMOVE BIBLIOGRAPHY**
\bibliography{auto_generated}   % will be created by the tdr script.

%
% Below is a multi-column format to display the standard PTDR symbols.
% Feel free to delete the lines from here to the end and the pdefs.tex file once you start modifying the template. The actual definitions are
% pulled in by the \def\Fileversion$#1: #2 ${\gdef\fileversion{#2}}
\def\Filedate$#1: #2-#3-#4 #5 ${\gdef\filedate{#2/#3/#4}}
\Fileversion$Revision: 237074 $
\Filedate$Date: 2014-04-17 15:17:42 +0200 (Thu, 17 Apr 2014) $
%%%%%%%%%%%%%%%%%%%%%%%%%%%%%%%%%%%%%%%%%%%%%%%%%%%%%%%%%%%%%%%%%%%%
%
%  CMS Common definitions style file
%
%  N.B. use of \newcommand rather than \newcommand means
%       that a definition is ignored if already specified
%
%                                              L. Taylor 18 Feb 2005
%%%%%%%%%%%%%%%%%%%%%%%%%%%%%%%%%%%%%%%%%%%%%%%%%%%%%%%%%%%%%%%%%%%%
\NeedsTeXFormat{LaTeX2e}
\ProvidesPackage{ptdr-definitions}[\filedate\space CMS Additional Physics Macros: dev version (\fileversion)]
\RequirePackage{heppennames2}
\RequirePackage{xspace}
\RequirePackage{amsmath}

% Some shorthand
% turn off italics
\newcommand {\etal}{\mbox{et al.}\xspace} %et al. - no preceding comma
\newcommand {\ie}{\mbox{i.e.}\xspace}     %i.e.
\newcommand {\eg}{\mbox{e.g.}\xspace}     %e.g.
\newcommand {\etc}{\mbox{etc.}\xspace}     %etc.
\newcommand {\vs}{\mbox{\sl vs.}\xspace}      %vs.
\newcommand {\mdash}{\ensuremath{\mathrm{-}}} % for use within formulas

% some terms whose definition we may change
\newcommand {\Lone}{Level-1\xspace} % Level-1 or L1 ?
\newcommand {\Ltwo}{Level-2\xspace}
\newcommand {\Lthree}{Level-3\xspace}

% Some software programs (alphabetized)
\newcommand{\ACERMC} {\textsc{AcerMC}\xspace}
\newcommand{\ALPGEN} {{\textsc{alpgen}}\xspace}
\newcommand{\CALCHEP} {{\textsc{CalcHEP}}\xspace}
\newcommand{\CHARYBDIS} {{\textsc{charybdis}}\xspace}
\newcommand{\CMKIN} {\textsc{cmkin}\xspace}
\newcommand{\CMSIM} {{\textsc{cmsim}}\xspace}
\newcommand{\CMSSW} {{\textsc{cmssw}}\xspace}
\newcommand{\COBRA} {{\textsc{cobra}}\xspace}
\newcommand{\COCOA} {{\textsc{cocoa}}\xspace}
\newcommand{\COMPHEP} {\textsc{CompHEP}\xspace}
\newcommand{\EVTGEN} {{\textsc{evtgen}}\xspace}
\newcommand{\FAMOS} {{\textsc{famos}}\xspace}
\newcommand{\GARCON} {\textsc{garcon}\xspace}
\newcommand{\GARFIELD} {{\textsc{garfield}}\xspace}
\newcommand{\GEANE} {{\textsc{geane}}\xspace}
\newcommand{\GEANTfour} {{\textsc{Geant4}}\xspace}
\newcommand{\GEANTthree} {{\textsc{geant3}}\xspace}
\newcommand{\GEANT} {{\textsc{geant}}\xspace}
\newcommand{\HDECAY} {\textsc{hdecay}\xspace}
\newcommand{\HERWIG} {{\textsc{herwig}}\xspace}
\newcommand{\HERWIGpp} {{\textsc{herwig++}}\xspace}
\newcommand{\POWHEG} {{\textsc{powheg}}\xspace}
\newcommand{\HIGLU} {{\textsc{higlu}}\xspace}
\newcommand{\HIJING} {{\textsc{hijing}}\xspace}
\newcommand{\IGUANA} {\textsc{iguana}\xspace}
\newcommand{\ISAJET} {{\textsc{isajet}}\xspace}
\newcommand{\ISAPYTHIA} {{\textsc{isapythia}}\xspace}
\newcommand{\ISASUGRA} {{\textsc{isasugra}}\xspace}
\newcommand{\ISASUSY} {{\textsc{isasusy}}\xspace}
\newcommand{\ISAWIG} {{\textsc{isawig}}\xspace}
\newcommand{\MADGRAPH} {\textsc{MadGraph}\xspace}
\newcommand{\MCATNLO} {\textsc{mc@nlo}\xspace}
\newcommand{\MCFM} {\textsc{mcfm}\xspace}
\newcommand{\MILLEPEDE} {{\textsc{millepede}}\xspace}
\newcommand{\ORCA} {{\textsc{orca}}\xspace}
\newcommand{\OSCAR} {{\textsc{oscar}}\xspace}
\newcommand{\PHOTOS} {\textsc{photos}\xspace}
\newcommand{\PROSPINO} {\textsc{prospino}\xspace}
\newcommand{\PYTHIA} {{\textsc{pythia}}\xspace}
\newcommand{\SHERPA} {{\textsc{sherpa}}\xspace}
\newcommand{\TAUOLA} {\textsc{tauola}\xspace}
\newcommand{\TOPREX} {\textsc{TopReX}\xspace}
\newcommand{\XDAQ} {{\textsc{xdaq}}\xspace}


%  Experiments
\newcommand {\DZERO}{D0\xspace}     %etc.


% Measurements and units...

\newcommand{\de}{\ensuremath{^\circ}}
\newcommand{\ten}[1]{\ensuremath{\times \text{10}^\text{#1}}}
\newcommand{\unit}[1]{\ensuremath{\text{\,#1}}\xspace}
\newcommand{\mum}{\ensuremath{\,\mu\text{m}}\xspace}
\newcommand{\micron}{\ensuremath{\,\mu\text{m}}\xspace}
\newcommand{\cm}{\ensuremath{\,\text{cm}}\xspace}
\newcommand{\mm}{\ensuremath{\,\text{mm}}\xspace}
\newcommand{\mus}{\ensuremath{\,\mu\text{s}}\xspace}
\newcommand{\keV}{\ensuremath{\,\text{ke\hspace{-.08em}V}}\xspace}
\newcommand{\MeV}{\ensuremath{\,\text{Me\hspace{-.08em}V}}\xspace}
\newcommand{\MeVns}{\ensuremath{\text{Me\hspace{-.08em}V}}\xspace} % no leading thinspace
\newcommand{\GeV}{\ensuremath{\,\text{Ge\hspace{-.08em}V}}\xspace}
\newcommand{\GeVns}{\ensuremath{\text{Ge\hspace{-.08em}V}}\xspace} % no leading thinspace
\newcommand{\gev}{\GeV}
\newcommand{\TeV}{\ensuremath{\,\text{Te\hspace{-.08em}V}}\xspace}
\newcommand{\TeVns}{\ensuremath{\text{Te\hspace{-.08em}V}}\xspace} % no leading thinspace
\newcommand{\PeV}{\ensuremath{\,\text{Pe\hspace{-.08em}V}}\xspace}
\newcommand{\keVc}{\ensuremath{{\,\text{ke\hspace{-.08em}V\hspace{-0.16em}/\hspace{-0.08em}}c}}\xspace}
\newcommand{\MeVc}{\ensuremath{{\,\text{Me\hspace{-.08em}V\hspace{-0.16em}/\hspace{-0.08em}}c}}\xspace}
\newcommand{\GeVc}{\ensuremath{{\,\text{Ge\hspace{-.08em}V\hspace{-0.16em}/\hspace{-0.08em}}c}}\xspace}
\newcommand{\GeVcns}{\ensuremath{{\text{Ge\hspace{-.08em}V\hspace{-0.16em}/\hspace{-0.08em}}c}}\xspace} % no leading thinspace
\newcommand{\TeVc}{\ensuremath{{\,\text{Te\hspace{-.08em}V\hspace{-0.16em}/\hspace{-0.08em}}c}}\xspace}
\newcommand{\keVcc}{\ensuremath{{\,\text{ke\hspace{-.08em}V\hspace{-0.16em}/\hspace{-0.08em}}c^\text{2}}}\xspace}
\newcommand{\MeVcc}{\ensuremath{{\,\text{Me\hspace{-.08em}V\hspace{-0.16em}/\hspace{-0.08em}}c^\text{2}}}\xspace}
\newcommand{\GeVcc}{\ensuremath{{\,\text{Ge\hspace{-.08em}V\hspace{-0.16em}/\hspace{-0.08em}}c^\text{2}}}\xspace}
\newcommand{\GeVccns}{\ensuremath{{\text{Ge\hspace{-.08em}V\hspace{-0.16em}/\hspace{-0.08em}}c^\text{2}}}\xspace} % no leading thinspace
\newcommand{\TeVcc}{\ensuremath{{\,\text{Te\hspace{-.08em}V\hspace{-0.16em}/\hspace{-0.08em}}c^\text{2}}}\xspace}

\newcommand{\pbinv} {\mbox{\ensuremath{\,\text{pb}^\text{$-$1}}}\xspace}
\newcommand{\fbinv} {\mbox{\ensuremath{\,\text{fb}^\text{$-$1}}}\xspace}
\newcommand{\nbinv} {\mbox{\ensuremath{\,\text{nb}^\text{$-$1}}}\xspace}
\newcommand{\mubinv} {\ensuremath{\,\mu\mathrm{b}^{-1}}\xspace}
\newcommand{\percms}{\ensuremath{\,\text{cm}^\text{$-$2}\,\text{s}^\text{$-$1}}\xspace}
\newcommand{\lumi}{\ensuremath{\mathcal{L}}\xspace}
\newcommand{\Lumi}{\ensuremath{\mathcal{L}}\xspace}%both upper and lower
%
% Need a convention here:
\newcommand{\LvLow}  {\ensuremath{\mathcal{L}=\text{10}^\text{32}\,\text{cm}^\text{$-$2}\,\text{s}^\text{$-$1}}\xspace}
\newcommand{\LLow}   {\ensuremath{\mathcal{L}=\text{10}^\text{33}\,\text{cm}^\text{$-$2}\,\text{s}^\text{$-$1}}\xspace}
\newcommand{\lowlumi}{\ensuremath{\mathcal{L}=\text{2}\times \text{10}^\text{33}\,\text{cm}^\text{$-$2}\,\text{s}^\text{$-$1}}\xspace}
\newcommand{\LMed}   {\ensuremath{\mathcal{L}=\text{2}\times \text{10}^\text{33}\,\text{cm}^\text{$-$2}\,\text{s}^\text{$-$1}}\xspace}
\newcommand{\LHigh}  {\ensuremath{\mathcal{L}=\text{10}^\text{34}\,\text{cm}^\text{$-$2}\,\text{s}^\text{$-$1}}\xspace}
\newcommand{\hilumi} {\ensuremath{\mathcal{L}=\text{10}^\text{34}\,\text{cm}^\text{$-$2}\,\text{s}^\text{$-$1}}\xspace}

% Physics symbols ...

\newcommand{\PT}{\ensuremath{p_{\mathrm{T}}}\xspace}
\newcommand{\pt}{\ensuremath{p_{\mathrm{T}}}\xspace}
\newcommand{\ET}{\ensuremath{E_{\mathrm{T}}}\xspace}
\newcommand{\HT}{\ensuremath{H_{\mathrm{T}}}\xspace}
\newcommand{\et}{\ensuremath{E_{\mathrm{T}}}\xspace}
\newcommand{\Em}{\ensuremath{E\hspace{-0.6em}/}\xspace}
\newcommand{\Pm}{\ensuremath{p\hspace{-0.5em}/}\xspace}
\newcommand{\PTm}{\ensuremath{{p}_\mathrm{T}\hspace{-1.02em}/\kern 0.5em}\xspace}
\newcommand{\PTslash}{\PTm}
\newcommand{\ETm}{\ensuremath{E_{\mathrm{T}}^{\text{miss}}}\xspace}
\newcommand{\MET}{\ETm}
\newcommand{\ETmiss}{\ETm}
\newcommand{\ETslash}{\ensuremath{E_{\mathrm{T}}\hspace{-1.1em}/\kern0.45em}\xspace}
\newcommand{\VEtmiss}{\ensuremath{{\vec E}_{\mathrm{T}}^{\text{miss}}}\xspace}
\newcommand{\ptvec}{\ensuremath{{\vec p}_{\mathrm{T}}}\xspace}

% roman face derivative
\newcommand{\dd}[2]{\ensuremath{\frac{\cmsSymbolFace{d} #1}{\cmsSymbolFace{d} #2}}}
\newcommand{\ddinline}[2]{\ensuremath{\cmsSymbolFace{d} #1/\cmsSymbolFace{d} #2}}
\newcommand{\rd}{\ensuremath{\cmsSymbolFace{d}}}
\newcommand{\re}{\ensuremath{\cmsSymbolFace{e}}}
% absolute value
\newcommand{\abs}[1]{\ensuremath{\lvert #1 \rvert}}



\ifthenelse{\boolean{cms@italic}}{\newcommand{\cmsSymbolFace}{\relax}}{\newcommand{\cmsSymbolFace}{\mathrm}}

% Particle names which track the italic/non-italic face convention
\newcommand{\zp}{{\PZpr}\xspace} % plain Z'
\newcommand{\JPsi}{{\PJGy}\xspace} % J/Psi (no mass)
\newcommand{\Z}{{\PZ}\xspace} % plain Z (no superscript 0)
\newcommand{\ttbar}{\PQt{}\PAQt\xspace} % t-tbar

% Extensions for missing names in PENNAMES % note no xspace, to match syntax in PENNAMES
\newcommand{\cPgn}{\PGn} % generic neutrino
\providecommand{\Pgn}{\PGn}
\newcommand{\cPagn}{\PAGn} % generic anti-neutrino
\providecommand{\Pagn}{\PAGn}
\newcommand{\cPgg}{\PGg} % gamma
\newcommand{\cPJgy}{\PJGy} % J/Psi (no mass)
\newcommand{\cPZ}{\PZ} % plain Z (no superscript 0)
\newcommand{\cPZpr}{\PZpr} % plain Z'
\newcommand{\cPqt}{\PQt} % t for t quark
\newcommand{\cPqb}{\PQb} % b for b quark
\newcommand{\cPqc}{\PQc} % c for c quark
\newcommand{\cPqs}{\PQs} % s for s quark
\newcommand{\cPqu}{\PQu} % u for u quark
\newcommand{\cPqd}{\PQd} % d for d quark
\newcommand{\cPq}{\PQq} % generic quark
\newcommand{\cPg}{\Pg} % generic gluon
\newcommand{\cPG}{\PXXG} % Graviton
\newcommand{\cPaqt}{\PAQt} % t for t anti-quark
\newcommand{\cPaqb}{\PAQb} % b for b anti-quark
\newcommand{\cPaqc}{\PAQc} % c for c anti-quark
\newcommand{\cPaqs}{\PAQs} % s for s anti-quark
\newcommand{\cPaqu}{\PAQu} % u for u anti-quark
\newcommand{\cPaqd}{\PAQd} % d for d anti-quark
\newcommand{\cPaq}{\PAQq} % generic anti-quark
\newcommand{\cPKstz}{\ensuremath{\cmsSymbolFace{K}^{\ast0}}\xspace}

% for APS style tables
\ifthenelse{\boolean{cms@external}}{%
\newenvironment{scotch}[1]{\protect\centering\ruledtabular\tabular{#1}}{\endtabular\endruledtabular}
}{
\newenvironment{scotch}[1]{\protect\centering\tabular{#1}\hline\hline}{\hline\endtabular}
}
% SM (still to be classified)

\newcommand{\AFB}{\ensuremath{A_\text{FB}}\xspace}
\newcommand{\wangle}{\ensuremath{\sin^{2}\theta_{\text{eff}}^\text{lept}(M^2_{\Z})}\xspace}
\newcommand{\stat}{\ensuremath{\,\text{(stat.)}}\xspace}
\newcommand{\syst}{\ensuremath{\,\text{(syst.)}}\xspace}
\newcommand{\lum}{\ensuremath{\,\text{(lum.)}}\xspace}
\newcommand{\kt}{\ensuremath{k_{\mathrm{T}}}\xspace}

\newcommand{\BC}{\HepParticle{B}{c}{}{}\xspace}
\newcommand{\bbarc}{\PAQb{}\PQc\xspace}
\newcommand{\bbbar}{\PQb{}\PAQb\xspace}
\newcommand{\ccbar}{\PQc{}\PAQc\xspace}
\newcommand{\bspsiphi}{\ensuremath{\PBs \to \JPsi\, \PGf}\xspace}
\newcommand{\EE}{\Pep{}\Pem\xspace}
\newcommand{\MM}{\PGmp{}\PGmm\xspace}
\newcommand{\TT}{\PGtm{}\PGtp\xspace}

%%%  E-gamma definitions
\newcommand{\HGG}{\ensuremath{\PH\to\PGg\PGg}}
\newcommand{\GAMJET}{\ensuremath{\PGg + \text{jet}}}
\newcommand{\PPTOJETS}{\ensuremath{\Pp\Pp\to\text{jets}}}
\newcommand{\PPTOGG}{\ensuremath{\Pp\Pp\to\PGg\PGg}}
\newcommand{\PPTOGAMJET}{\ensuremath{\Pp\Pp\to\PGg + \text{jet}}}
\newcommand{\MH}{\ensuremath{M_{\PH}}}
\newcommand{\RNINE}{\ensuremath{R_\mathrm{9}}}
\newcommand{\DR}{\ensuremath{\Delta R}}





%%%%%%
% From Albert
%

\newcommand{\ga}{\ensuremath{\gtrsim}}
\newcommand{\la}{\ensuremath{\lesssim}}
%
\newcommand{\swsq}{\ensuremath{\sin^2\theta_{\PW}}\xspace}
\newcommand{\cwsq}{\ensuremath{\cos^2\theta_{\PW}}\xspace}
\newcommand{\tanb}{\ensuremath{\tan\beta}\xspace}
\newcommand{\tanbsq}{\ensuremath{\tan^{2}\beta}\xspace}
\newcommand{\sidb}{\ensuremath{\sin 2\beta}\xspace}
\newcommand{\alpS}{\ensuremath{\alpha_S}\xspace}
\newcommand{\alpt}{\ensuremath{\widetilde{\alpha}}\xspace}

\newcommand{\QL}{\HepParticle{Q}{L}{}{}\xspace}
\newcommand{\sQ}{\HepSusyParticle{Q}{}{}{}\xspace}
\newcommand{\sQL}{\HepSusyParticle{Q}{L}{}{}\xspace}
\newcommand{\ULC}{\HepParticle{U}{L}{C}{}\xspace}
\newcommand{\sUC}{\HepSusyParticle{U}{}{C}{}\xspace}
\newcommand{\sULC}{\HepSusyParticle{U}{L}{C}{}\xspace}
\newcommand{\DLC}{\HepParticle{D}{L}{C}{}\xspace}
\newcommand{\sDC}{\HepSusyParticle{D}{}{C}{}\xspace}
\newcommand{\sDLC}{\HepSusyParticle{D}{L}{C}{}\xspace}
\newcommand{\LL}{\HepParticle{L}{L}{}{}\xspace}
\newcommand{\sL}{\HepSusyParticle{L}{}{}{}\xspace}
\newcommand{\sLL}{\HepSusyParticle{L}{L}{}{}\xspace}
\newcommand{\ELC}{\HepParticle{E}{L}{C}{}\xspace}
\newcommand{\sEC}{\HepSusyParticle{E}{}{C}{}\xspace}
\newcommand{\sELC}{\HepSusyParticle{E}{L}{C}{}\xspace}
\newcommand{\sEL}{\HepSusyParticle{E}{L}{}{}\xspace}
\newcommand{\sER}{\HepSusyParticle{E}{R}{}{}\xspace}
\newcommand{\sFer}{\HepSusyParticle{f}{}{}{}\xspace}
\newcommand{\sQua}{\PSQ}
\newcommand{\sUp}{\PSQu}
\newcommand{\suL}{\PSQuL}
\newcommand{\suR}{\PSQuR}
\newcommand{\sDw}{\PSQd}
\newcommand{\sdL}{\PSQdL}
\newcommand{\sdR}{\PSQdR}
\newcommand{\sTop}{\PSQt}
\newcommand{\stL}{\PSQtL}
\newcommand{\stR}{\PSQtR}
\newcommand{\stone}{\PSQtDo}
\newcommand{\sttwo}{\PSQtDt}
\newcommand{\sBot}{\PSQb}
\newcommand{\sbL}{\PSQbL}
\newcommand{\sbR}{\PSQbR}
\newcommand{\sbone}{\PSQbDo}
\newcommand{\sbtwo}{\PSQbDt}
\newcommand{\sLep}{\PSl}
\newcommand{\sLepC}{\HepSusyParticle{l}{}{C}{}\xspace}
\newcommand{\sEl}{\PSe}
\newcommand{\sElC}{\HepSusyParticle{e}{}{C}{}\xspace}
\newcommand{\seL}{\PSeL}
\newcommand{\seR}{\PSeR}
\newcommand{\snL}{\HepSusyParticle{\PGn}{L}{}{}\xspace}
\newcommand{\sMu}{\PSGm}
\newcommand{\sNu}{\PSGn}
\newcommand{\sTau}{\PSGt}
\newcommand{\Glu}{\Pg}
\newcommand{\sGlu}{\PSg}
\newcommand{\Wpm}{\PWpm}
\newcommand{\sWpm}{\PSWpm}
\newcommand{\Wz}{\HepParticle{W}{}{0}{}\xspace}
\newcommand{\sWz}{\HepSusyParticle{W}{}{0}{}\xspace}
\newcommand{\sWino}{\PSW}
\newcommand{\Bz}{\HepParticle{B}{}{0}\xspace}
\newcommand{\sBz}{\HepSusyParticle{B}{}{0}\xspace}
\newcommand{\sBino}{\HepSusyParticle{B}{}{}\xspace}
\newcommand{\Zz}{\PZz}
\newcommand{\sZino}{\PSZz}
\newcommand{\sGam}{\PSGg}
\newcommand{\chiz}{\PSGcz}
\newcommand{\chip}{\PSGcp}
\newcommand{\chim}{\PSGcm}
\newcommand{\chipm}{\PSGcpm}
\newcommand{\Hone}{\HepParticle{H}{d}{}\xspace}
\newcommand{\sHone}{\HepSusyParticle{H}{d}{}\xspace}
\newcommand{\Htwo}{\HepParticle{H}{u}{}\xspace}
\newcommand{\sHtwo}{\HepSusyParticle{H}{u}{}\xspace}
\newcommand{\sHig}{\HepSusyParticle{H}{}{}{}\xspace}
\newcommand{\sHa}{\HepSusyParticle{H}{a}{}{}\xspace}
\newcommand{\sHb}{\HepSusyParticle{H}{b}{}{}\xspace}
\newcommand{\sHpm}{\HepSusyParticle{H}{}{\pm}{}\xspace}
\newcommand{\hz}{\PShz}
\newcommand{\Hz}{\PHz}
\newcommand{\Az}{\PSAz}
\newcommand{\Hpm}{\PSHpm}
\newcommand{\sGra}{\PXXSG}
%
\newcommand{\mtil}{\ensuremath{\widetilde{m}}\xspace}
%
\newcommand{\rpv}{\ensuremath{\rlap{\kern.2em/}R}\xspace}
\newcommand{\LLE}{\ensuremath{LL\bar{E}}\xspace}
\newcommand{\LQD}{\ensuremath{LQ\bar{D}}\xspace}
\newcommand{\UDD}{\ensuremath{\overline{UDD}}\xspace}
\newcommand{\Lam}{\ensuremath{\lambda}\xspace}
\newcommand{\Lamp}{\ensuremath{\lambda'}\xspace}
\newcommand{\Lampp}{\ensuremath{\lambda''}\xspace}
%
\newcommand{\spinbd}[2]{\ensuremath{\bar{#1}_{\dot{#2}}}\xspace}

\newcommand{\MD}{\ensuremath{{M_\mathrm{D}}}\xspace}% ED mass
\newcommand{\Mpl}{\ensuremath{{M_\mathrm{Pl}}}\xspace}% Planck mass
\newcommand{\Rinv} {\ensuremath{{R}^{-1}}\xspace}

%pennames to heppenames2 translation
%no longer include mass
%PDstpm
%PKst
%
%no j subscript, no tilde
%PSHpm
%PSHz
%
%Added overbar
%PagXz
%
%Removed overbar
%PgXz
%
\newcommand{\PAz}{\PSAz}
\newcommand{\PDiz}{\PDzDoP{2420}}
\newcommand{\PDstiiz}{\PDstzDtP{2460}}
\newcommand{\PDstz}{\PDstzP{2010}}
\newcommand{\PHpm}{\PSHpm}
\newcommand{\PJgy}{\PJGyP{1S}}
\newcommand{\PKia}{\PKDoP{1400}}
\newcommand{\PKii}{\PKDtP{1770}}
\newcommand{\PKi}{\PKDoP{1270}}
\newcommand{\PKsta}{\PKstP{1370}}
\newcommand{\PKstb}{\PKstP{1680}}
\newcommand{\PKstiii}{\PKstDTP{1780}}
\newcommand{\PKstii}{\PKstDtP{1430}}
\newcommand{\PKstiv}{\PKstDfP{2045}}
\newcommand{\PKstz}{\PKstDzP{1430}}
\newcommand{\PNa}{\HepParticleResonanceFormalFull{\PN}{}{}{1440}{}{}{P}{11}{}\xspace}
\newcommand{\PNb}{\HepParticleResonanceFormalFull{\PN}{}{}{1520}{}{}{D}{13}{}\xspace}
\newcommand{\PNc}{\HepParticleResonanceFormalFull{\PN}{}{}{1535}{}{}{S}{11}{}\xspace}
\newcommand{\PNd}{\HepParticleResonanceFormalFull{\PN}{}{}{1650}{}{}{S}{11}{}\xspace}
\newcommand{\PNe}{\HepParticleResonanceFormalFull{\PN}{}{}{1675}{}{}{D}{15}{}\xspace}
\newcommand{\PNf}{\HepParticleResonanceFormalFull{\PN}{}{}{1680}{}{}{F}{15}{}\xspace}
\newcommand{\PNg}{\HepParticleResonanceFormalFull{\PN}{}{}{1700}{}{}{D}{13}{}\xspace}
\newcommand{\PNh}{\HepParticleResonanceFormalFull{\PN}{}{}{1710}{}{}{P}{11}{}\xspace}
\newcommand{\PNi}{\HepParticleResonanceFormalFull{\PN}{}{}{1720}{}{}{P}{13}{}\xspace}
\newcommand{\PNj}{\HepParticleResonanceFormalFull{\PN}{}{}{2190}{}{}{G}{17}{}\xspace}
\newcommand{\PNk}{\HepParticleResonanceFormalFull{\PN}{}{}{2220}{}{}{H}{19}{}\xspace}
\newcommand{\PNl}{\HepParticleResonanceFormalFull{\PN}{}{}{2250}{}{}{G}{19}{}\xspace}
\newcommand{\PNm}{\HepParticleResonanceFormalFull{\PN}{}{}{2600}{}{}{I}{1,11}{}\xspace}
\newcommand{\PSHz}{\HepParticle{\PSH}{}{0}{}}
\newcommand{\PSgg}{\PSGg}
\newcommand{\PSgm}{\PSGm}
\newcommand{\PSgn}{\PSGn}
\newcommand{\PSgt}{\PSGt}
\newcommand{\PSgxpm}{\PSGcpm}
\newcommand{\PSgxz}{\PSGcz}
\newcommand{\PSq}{\PSQ}
\newcommand{\PZgc}{\PZGc}
\newcommand{\PZge}{\PZGe}
\newcommand{\PZgy}{\PZGy}
\newcommand{\PZi}{\HepParticle{Z}{1}{}\xspace}
\newcommand{\PaBz}{\PABz}
\newcommand{\PaB}{\PAB}
\newcommand{\PaDz}{\PADz}
\newcommand{\PaD}{\PAD}
\newcommand{\PaKz}{\PAKz}
\newcommand{\PaSq}{\PASQ}
\newcommand{\PagL}{\PAGL}
\newcommand{\PagOp}{\HepAntiParticle{\PGO}{}{+}\xspace}
\newcommand{\PagSm}{\HepAntiParticle{\PGS}{}{-}\xspace}
\newcommand{\PagSp}{\HepAntiParticle{\PGS}{}{+}\xspace}
\newcommand{\PagSz}{\HepAntiParticle{\PGS}{}{0}\xspace}
\newcommand{\PagXp}{\HepAntiParticle{\PGX}{}{+}\xspace}
\newcommand{\PagXz}{\HepAntiParticle{\PGX}{}{0}\xspace}
\newcommand{\Pagne}{\PAGne}
\newcommand{\Pagngm}{\PAGnGm}
\newcommand{\Pagngt}{\PAGnGt}
\newcommand{\Paii}{\PaDtP{1320}}
\newcommand{\Pai}{\PaDoP{1260}}
\newcommand{\Pap}{\PAp}
\newcommand{\Paqb}{\PAQqb}
\newcommand{\Paqc}{\PAQqc}
\newcommand{\Paqd}{\PAQqd}
\newcommand{\Paqs}{\PAQqs}
\newcommand{\Paqt}{\PAQqt}
\newcommand{\Paqu}{\PAQqu}
\newcommand{\Paq}{\PAQq}
\newcommand{\Paz}{\PaDzP{980}}
\newcommand{\Pbgcia}{\PGcbDoP{2P}}
\newcommand{\Pbgciia}{\PGcbDtP{2P}}
\newcommand{\Pbgcii}{\PGcbDtP{1P}}
\newcommand{\Pbgci}{\PGcbDoP{1P}}
\newcommand{\Pbgcza}{\PGcbDzP{2P}}
\newcommand{\Pbgcz}{\PGcbDzP{1P}}
\newcommand{\Pbi}{\PbDoP{1235}}
\newcommand{\PcgLp}{\PGLpc}
\newcommand{\PcgS}{\PGScP{2455}}
\newcommand{\PcgXp}{\PGXpc}
\newcommand{\PcgXz}{\PGXzc}
\newcommand{\Pcgcii}{\PGccDtP{1P}}
\newcommand{\Pcgci}{\PGccDoP{1P}}
\newcommand{\Pcgcz}{\PGccDzP{1S}}
\newcommand{\Pcgh}{\PGhcP{1S}}
\newcommand{\Pfia}{\PfDoP{1390}}
\newcommand{\Pfib}{\PfDoP{1510}}
\newcommand{\Pfiia}{\PfDtP{1720}}
\newcommand{\Pfiib}{\PfDtP{2010}}
\newcommand{\Pfiic}{\PfDtP{2300}}
\newcommand{\Pfiid}{\PfDtP{2340}}
\newcommand{\Pfiipr}{\PfprDtP{1525}}
\newcommand{\Pfii}{\PfDtP{1270}}
\newcommand{\Pfiv}{\PfDfP{2050}}
\newcommand{\Pfi}{\PfDoP{1285}}
\newcommand{\Pfza}{\PfDzP{1400}}
\newcommand{\Pfzb}{\PfDzP{1590}}
\newcommand{\Pfz}{\PfDzP{975}}
\newcommand{\PgD}{\PGD}
\newcommand{\PgDa}{\HepParticleResonanceFormalFull{\PGD}{}{}{1232}{}{}{P}{33}{}\xspace}
\newcommand{\PgDb}{\HepParticleResonanceFormalFull{\PGD}{}{}{1620}{}{}{S}{31}{}\xspace}
\newcommand{\PgDc}{\HepParticleResonanceFormalFull{\PGD}{}{}{1700}{}{}{D}{33}{}\xspace}
\newcommand{\PgDd}{\HepParticleResonanceFormalFull{\PGD}{}{}{1900}{}{}{S}{31}{}\xspace}
\newcommand{\PgDe}{\HepParticleResonanceFormalFull{\PGD}{}{}{1905}{}{}{F}{35}{}\xspace}
\newcommand{\PgDf}{\HepParticleResonanceFormalFull{\PGD}{}{}{1910}{}{}{P}{31}{}\xspace}
\newcommand{\PgDh}{\HepParticleResonanceFormalFull{\PGD}{}{}{1920}{}{}{P}{33}{}\xspace}
\newcommand{\PgDi}{\HepParticleResonanceFormalFull{\PGD}{}{}{1930}{}{}{D}{35}{}\xspace}
\newcommand{\PgDj}{\HepParticleResonanceFormalFull{\PGD}{}{}{1950}{}{}{F}{37}{}\xspace}
\newcommand{\PgDk}{\HepParticleResonanceFormalFull{\PGD}{}{}{2420}{}{}{H}{3,11}{}\xspace}
\newcommand{\PgL}{\PGL}
\newcommand{\PgLa}{\HepParticleResonanceFormalFull{\PGL}{}{}{1405}{}{}{S}{01}{}}
\newcommand{\PgLb}{\HepParticleResonanceFormalFull{\PGL}{}{}{1520}{}{}{D}{03}{}}
\newcommand{\PgLc}{\HepParticleResonanceFormalFull{\PGL}{}{}{1600}{}{}{P}{01}{}}
\newcommand{\PgLd}{\HepParticleResonanceFormalFull{\PGL}{}{}{1670}{}{}{S}{01}{}}
\newcommand{\PgLe}{\HepParticleResonanceFormalFull{\PGL}{}{}{1690}{}{}{D}{03}{}}
\newcommand{\PgLf}{\HepParticleResonanceFormalFull{\PGL}{}{}{1800}{}{}{S}{01}{}}
\newcommand{\PgLg}{\HepParticleResonanceFormalFull{\PGL}{}{}{1810}{}{}{P}{01}{}}
\newcommand{\PgLh}{\HepParticleResonanceFormalFull{\PGL}{}{}{1820}{}{}{F}{05}{}}
\newcommand{\PgLi}{\HepParticleResonanceFormalFull{\PGL}{}{}{1830}{}{}{D}{05}{}}
\newcommand{\PgLj}{\HepParticleResonanceFormalFull{\PGL}{}{}{1890}{}{}{P}{03}{}}
\newcommand{\PgLk}{\HepParticleResonanceFormalFull{\PGL}{}{}{2100}{}{}{G}{07}{}}
\newcommand{\PgLl}{\HepParticleResonanceFormalFull{\PGL}{}{}{2110}{}{}{F}{05}{}}
\newcommand{\PgLm}{\HepParticleResonanceFormalFull{\PGL}{}{}{2350}{}{}{H}{09}{}}
\newcommand{\PgO}{\PGO}
\newcommand{\PgOm}{\PGOm}
\newcommand{\PgOma}{\PGOmP{2250}}
\newcommand{\PgS}{\PGS}
\newcommand{\PgSa}{\HepParticleResonanceFormalFull{\PGS}{}{}{1385}{}{}{P}{13}{}\xspace}
\newcommand{\PgSb}{\HepParticleResonanceFormalFull{\PGS}{}{}{1660}{}{}{P}{11}{}\xspace}
\newcommand{\PgSc}{\HepParticleResonanceFormalFull{\PGS}{}{}{1670}{}{}{D}{13}{}\xspace}
\newcommand{\PgSd}{\HepParticleResonanceFormalFull{\PGS}{}{}{1750}{}{}{S}{11}{}\xspace}
\newcommand{\PgSe}{\HepParticleResonanceFormalFull{\PGS}{}{}{1775}{}{}{D}{15}{}\xspace}
\newcommand{\PgSf}{\HepParticleResonanceFormalFull{\PGS}{}{}{1915}{}{}{F}{15}{}\xspace}
\newcommand{\PgSg}{\HepParticleResonanceFormalFull{\PGS}{}{}{1940}{}{}{D}{13}{}\xspace}
\newcommand{\PgSh}{\HepParticleResonanceFormalFull{\PGS}{}{}{2030}{}{}{F}{17}{}\xspace}
\newcommand{\PgSi}{\PGSP{2050}}
\newcommand{\PgSm}{\PGSm}
\newcommand{\PgSp}{\PGSp}
\newcommand{\PgSz}{\PGSz}
\newcommand{\PgU}{\PGU}
\newcommand{\PgUa}{\PGUP{1S}}
\newcommand{\PgUb}{\PGUP{2S}}
\newcommand{\PgUc}{\PGUP{3S}}
\newcommand{\PgUd}{\PGUP{4S}}
\newcommand{\PgUe}{\PGUP{10860}}
\newcommand{\PgUf}{\PGUP{11020}}
\newcommand{\PgX}{\PGX}
\newcommand{\PgXa}{\HepParticleResonanceFormalFull{\PGX}{}{}{1530}{}{}{P}{13}{}\xspace}
\newcommand{\PgXb}{\PGXP{1690}}
\newcommand{\PgXc}{\HepParticleResonanceFormalFull{\PGX}{}{}{1820}{}{}{D}{13}{}\xspace}
\newcommand{\PgXd}{\PGXP{1950}}
\newcommand{\PgXe}{\PGXP{2030}}
\newcommand{\PgXm}{\PGXm}
\newcommand{\PgXz}{\HepParticle{\PGX}{}{0}{}}
\newcommand{\Pgfa}{\PGfP{1680}}
\newcommand{\Pgfiii}{\PGfDTP{1850}}
\newcommand{\Pgf}{\PGfP{1020}}
\newcommand{\Pgg}{\PGg}
\newcommand{\Pgha}{\PGhP{1295}}
\newcommand{\Pghb}{\PGhP{1440}}
\newcommand{\Pghpr}{\PGhprP{958}}
\newcommand{\Pgh}{\PGh}
\newcommand{\Pgmm}{\PGmm}
\newcommand{\Pgmp}{\PGmp}
\newcommand{\Pgm}{\PGm}
\newcommand{\Pgne}{\PGne}
\newcommand{\Pgngm}{\PGnGm}
\newcommand{\Pgngt}{\PGnGt}
\newcommand{\Pgoa}{\PGoP{1390}}
\newcommand{\Pgob}{\PGoP{1600}}
\newcommand{\Pgoiii}{\PGoDTP{1670}}
\newcommand{\Pgo}{\PGoP{783}}
\newcommand{\Pgpa}{\PGpP{1300}}
\newcommand{\Pgpii}{\PGpDtP{1670}}
\newcommand{\Pgpm}{\PGpm}
\newcommand{\Pgppm}{\PGppm}
\newcommand{\Pgpp}{\PGpp}
\newcommand{\Pgpz}{\PGpz}
\newcommand{\Pgp}{\PGp}
\newcommand{\Pgra}{\PGrP{1450}}
\newcommand{\Pgrb}{\PGrP{1700}}
\newcommand{\Pgriii}{\PGrDTP{1690}}
\newcommand{\Pgr}{\PGrP{770}}
\newcommand{\Pgt}{\PGt}
\newcommand{\Pgya}{\PGyP{3770}}
\newcommand{\Pgyb}{\PGyP{4040}}
\newcommand{\Pgyc}{\PGyP{4160}}
\newcommand{\Pgyd}{\PGyP{4415}}
\newcommand{\Pgy}{\PGyP{2S}}
\newcommand{\Phia}{\PhDoP{1170}}
\newcommand{\Pqb}{\PQqb}
\newcommand{\Pqc}{\PQqc}
\newcommand{\Pqd}{\PQqd}
\newcommand{\Pqs}{\PQqs}
\newcommand{\Pqt}{\PQqt}
\newcommand{\Pqu}{\PQqu}
\newcommand{\Pq}{\PQq}
\newcommand{\PsDipm}{\PDpmsDoP{2536}}
\newcommand{\PsDm}{\PDms}
\newcommand{\PsDp}{\PDps}
\newcommand{\PsDst}{\HepParticle{\PD}{s}{\ast}\xspace}
%
\endinput
 at the beginning.
%
